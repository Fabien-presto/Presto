% This file is part of Presto-HF, a matlab toolbox to identify a circuit from
% its response.
%
% SPDX-License-Identifier: AGPL-3.0-or-later
%
% Copyright 2025 by
%   Centre Inria de l'Université Côte d'Azur
%   2004, route des Lucioles
%   06902 Sophia Antipolis Cedex
%
% and by
%   Mines Paris - Armines
%   60, boulevard Saint-Michel
%   75006 Paris
%
% Contributors: Fabien Seyfert, Jean-Paul Marmorat, Martine Olivi
%
% Presto-HF is free software: you can redistribute it and/or modify it under
% the terms of the GNU Affero General Public License as published by the Free
% Software Foundation, either version 3 of the License, or (at your option) any
% later version.
%
% Presto-HF is distributed in the hope that it will be useful, but WITHOUT ANY
% WARRANTY; without even the implied warranty of MERCHANTABILITY or FITNESS FOR
% A PARTICULAR PURPOSE. See the GNU Affero General Public License for more
% details.
%
% You should have received a copy of the GNU Affero General Public License
% along with Presto-HF. If not, see <https://www.gnu.org/licenses/>.
%
\label{Comp}
\FuncDef{[So]=CompensateDelayAndFreqShift(S,zeros,plotflag)}
{\fitem[So] S enriched with informations about completion and
compensation. The frequency data of So are compensated}
{%
\fitem[S] The mesured data to be compensated, contained in the fields
{\it freq} and {\it value} of the S-strucutre S.
\fitem[zeros=DVC.CDAFS.zeros\_at\_inf] Number of zeros at infty for
$S_{1,2}$ and $S_{2,1}$. This information is used during the computation of 
a completion for those elements.
\fitem[plotflag=DVC.CDAFS.plot\_flag] If plotflag is diff of 0 the
result is ploted}
{%
\fitem[DVC.CDAFS.sign\_in\_zero\_trans\_12] The constant frequency shift applied to
$S_{1,2}$ and $S_{2,1}$ is defined up to $\pi$. This constant fixes the
sign of the imaginary part of $S_{1,2}$ at the zero frequency (at the
measurment point nearest to zeros). Poss. values \{-1,1\}. {\bf
Important}: the identified coupling matrices corresponding to those
nomalizations can be deduced from one an other by a change of sign of
certain couplings.   
\fitem[DVC.CDAFS.delay\_range] Range of possible delay compensation to
be considered. 
\fitem[DVC.CDAFS.poly\_order] Degree of the $1/s$ polynomial used to
compute completion.
\fitem[DVC.CDAFS.omega\_lim] For $|w|>w_{lim}$ the $1/s$ polynomial
defining the completion, should fit the data as good as possible. In
other words the users believes that for $|w|>w_{lim}$ a $1/s$ expansion
of degree DVC.CDAFS.poly\_order is a good approximation of the frequency 
behaviour of the filter. {\bf Important}: the number of selected data
points should be enough so that a meaningfull approximation can be
computed, i.e $card(|w_k|>w_{lim})>>poly\_order$.
\fitem[DVC.CDAFS.error\_lim] If the error between the data (with best
compensation) and the $1/s$ expansion (on the data points with $|w|>w\_lim$) 
is greater than error\_lim a warning message is released.
\fitem[DVC.CDAFS.causal\_bound(1,1)] Bound for the distance to causal
systems for the completion of $S_{1,1}$. This bound is expressed in
percent.
\fitem[DVC.CDAFS.causal\_bound(2,2)] Same as prev. but for $S_{2,2}$.
\fitem[DVC.CDAFS.modulus\_factor(1,1)] The modulus of the completion is
controled so as to be less than: $modulus\_factor * max_{w}(|S_{1,1}|)$.
\fitem[DVC.CDAFS.modulus\_factor(2,2)] Same as prev. but for $S_{2,2}$.
\fitem[DVC.CDAFS.number\_of\_control\_points] Number of (equaly spaced
on the sub-arc of the unit circle where the completion is defined) control points used to control the modulus of the completion. This number should be choosen
in regards of DVC.CDAFS.poly\_order (good results seems to be obtained
with $number\_of\_control\_points \geq 5*.poly\_order$).
\fitem[DVC.CDAFS.number\_of\_fourier\_coeffs] Number of Fourier
coeffs. used to estimate non-causal part of completed data.
\fitem[DVC.CDAFS.iso\_flag] Indicates which kind of isometry is used
when passing to the disk to evaluate the non-causal part of completed
data. Possibilities are $L^{\infty}$ or $L^2$ isometry which are
obtained resp. for the values \{0,1\} of the flag.   
}
{
The function proceeds roughly in three steps: first the delay components
of $S_{1,1}$ and $S_{2,2}$ are estimated and compensated, then completions are
determined for all entries and finally constant frequency shifts are
applied so as to ensure $Imag(S_{1,1})=Imag(S_{2,2})=0$. A frequency
shifts on the diagonal terms is also estimated by using the hypothesis
that in the ideal case the system is lossless.   
\\
{\bf Delay determination:}
For $S_{1,1}$ and $S_{2,2}$ the delay component is estimated in the
following way. Let $K=\{k,\,|w_k|>w_{lim}$ the selected measurement
points. Given $\tau \in [delay\_range]$ we define 
$$ \psi(\tau)=\min_{(a_0,a_1...) \in \Bbb{C}^{poly\_order+1}} \sum_{k\in K}
|S_{1,1}(w_k)e^{i\tau w_k} - \sum_{l=0}^{poly\_order}
\frac{a_l}{(w_k)^l}|^2.
$$ The 'optimal' delay compensation $\tau_{opt}$ is defined by, 
$$\psi(\tau_{opt})=\min_{\tau \in [delay\_range]} \psi(\tau) $$       

{\bf Completions:}
We recall that $I_2$ and $I_{\infty}$ are respectively the $L^2$ and
$L^{\infty}$ isometries definied as in \ref{isom}. Let $p$ be a
polynomial defining a possible completion of the measured data we denote 
by $S_{i,j} \vee p$ the extended function defined on the whole imaginary axes 
(note that between two measurement points, $S_{i,j}$ is defined by
spline interpolation, so that it can be considered as a function on the
interval $[\min(w_i),\max(w_i)]$). The constrained optimisation problem
we adress to determine a completion of our data (for $iso\_flag=0$) is the following:
\begin{align*}
& !\min_{p \in \Bbb{C}^{poly\_order}[x]}\sum_{k \in
K}|S_{i,j}(w_k)-p(\frac{1}{w_k})|^2 \\
\intertext{under the following constraints}
& ||P_{H^2_0}(I_{\infty}(S_{i,j}) \vee p)||_2 \leq ||I_{\infty}(S_{i,j})||_2
.causal\_bound(i,j) \\
&\forall w \in [control\_points] \quad|p(\frac{1}{w_c})| \leq
modulus\_factor(i,j).\max_k(|S_{i,j}(w_k)|) 
\end{align*}
If there is no element $p$ satisfying those constraints then the
following relaxed problem is solved,
\begin{align*}
& !\min_{p \in \Bbb{C}^{poly\_order}[x]}  ||P_{H^2_0}(I_{\infty}(S_{i,j}) \vee p)||_2^2 
\intertext{under the following constraint}
&\forall w \in [control\_points] \quad |p(\frac{1}{w_c})| \leq
modulus\_factor(i,j).\max_k(|S_{i,j}(w_k)|) 
\end{align*}
for which $p=0$ is a feasable point. For $iso\_flag=1$ the $L^2$ isometry 
is used instead (the value at infinity is substracted, so as to ensure
that $S_{i,j} \vee p - p_0$ is $L^2$ integrable).
}
     

